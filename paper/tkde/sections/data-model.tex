\subsection{Data Model}
Here we discuss the data model for observational data managed by Synopsis. Individual observations are geotagged and have chronological timestamps indicating where and when the observations were made. Location information is encoded as a $\langle latitude, longitude \rangle$ tuple. Observations are multidimensional with multiple features (e.g, temperature, humidity, wind speed, etc.) being reported in each observation. These features may be encoded as $\langle feature\_name, value \rangle$ tuples or may have predefined positions with the observational data's serialized representation. 

%TODO: does this add anything to the discussion? It seemed a bit out of place to me and you could say the same thing about other summary statistics maintained by the system (that they are not already encoded in the data) [malensek].
%Features encapsulated within these observations may be correlated.  Pairwise correlation coefficients are not encoded within the observations, but must instead be computed from the data. Observations are reported independently of each other, but for a given feature, values reported in successive observations may be correlated; once again, autocorrelation across successive measurements are not encoded in the observations. 

\textbf{SIFT:} The SIFT data structure is organized as a forest of trees. Data encapsulated within the observations are used to populate trees within SIFT. Each tree within SIFT has the geocoding information as its root node, temporal information at the second level, and individual features encapsulated within the observations at successive levels. The system may shuffle positioning of the features to conserve memory or speed up evaluations for dominant queries.

Additional nodes may be introduced to the trees within a SIFT to collate observations that fall within a certain range. In the case of chronological information these nodes correspond to different temporal scopes such as hours, days, weeks, months, and years. In the case of individual features, these nodes represent a binning of the feature values that correspond to the density in the distribution of observed feature values. Each feature node stores auxiliary information such as min, mean, max, standard deviation, and number of observations to capture statistical aspects about the distribution of values within the bin. Summary statistics are updated in an online fashion and keep pace with data arrival rates.

\textbf{Systems View of the Sketch:} The Synopsis sketch, comprising sketchlets dispersed over multiple machines, is a compact and memory-resident surrogate for the entire observational space. The sketch may be used for any purpose that regular, on-disk data is used for including but not limited to: query evaluations, assessing statistical properties of the data, and launching custom computations using well-known analytical engines. The system imposes no restrictions on the programming language or the type of processing logic that may be encoded within the computations. 

The Synopsis sketch is adaptive and evolves over time. The number of skethlets comprising the sketch varies as the sketch performs scaling maneuvers to cope with data volumes and memory management. The number of trees within the SIFT  also varies over time as temporal scopes are aggregated, features binned, and sketchlets are spawned and fused during scaling operations.
