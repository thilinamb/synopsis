\documentclass{article}
\usepackage[utf8]{inputenc}
\usepackage{graphicx}
\usepackage{geometry}
\geometry{
    letterpaper,
    top=1in, bottom=1in,
    left=1in, right=1in,
}
\usepackage{times}

\usepackage{color}
\usepackage[most]{tcolorbox}

\definecolor{lightblue}{HTML}{e9f3f9}
\definecolor{darkblue}{HTML}{5bacde}
\definecolor{shadowcolor}{HTML}{cecece}
\tcbset{
    skin=enhanced,
    drop shadow=shadowcolor,
    left=0.5em,
    right=0.5em,
    top=0.5em,
    bottom=0.5em,
    before skip=1.5em,
    after skip=1.5em,
    colback=lightblue,
    colframe=darkblue,
    boxrule=0.75pt,
    parbox=false
}

\begin{document}

\section*{Summary of Changes}\label{summary-of-changes}
\subsection*{Editor Comments}\label{editor-comments}

We have received three review reports. While all the reviewers
appreciate your research efforts, a number of concerns on the novelty,
performance bound, query accuracy, experiments, etc. have been raised.
Please address all these concerns in a major revision.

\begin{tcolorbox}
To begin, we would like to thank the editors and reviewers for their
valuable feedback. We were pleased to receive and implement these
suggestions, and believe that the revised manuscript has been improved
substantially as a result. Our main improvements to the text include:

\begin{itemize}
\item
  Edits that address each of the questions raised by the reviewers
  (explained in greater detail below), along with substantial cleanup
  and formatting tweaks to improve readability
\item
  A revamped contributions section in the introduction to highlight the
  novelty of our SIFT data structure
\item
  A memory growth benchmark that incorporates a second geospatial
  dataset. This illustrates that the memory consumption profile observed
  with our first dataset applies to another geospatial dataset as well.
\item
  Pseudocode representation of the SIFT structural compaction algorithm
  and calculation of the fan-out score
\item
  A benchmark comparing the query performance of Spark SQL with Synopsis
  to demonstrate the benefit of the data structures and systems design
  employed by Synopsis
\item
  Additions to the RF benchmark (?)
\end{itemize}
\end{tcolorbox}

\subsection*{Reviewer: 1}\label{reviewer-1}

This paper proposes a novel distributed sketch, Synopsis, to index
spatiotemporal stream data. Synopsis is update friendly. Specifically,
Synopsis can scale effectively when the arrive rate of the data stream
is faster than the rate at which the Synopsis can be updated. Synopsis
is a summary of the data but most information is reserved such that
queries on Synopsis can produce results of high accuracy.

\begin{tcolorbox}
Thank you for your review; we have addressed each of the points below
and feel that these suggestions were extremely helpful in improving our
work!
\end{tcolorbox}

\begin{enumerate}
\def\labelenumi{\arabic{enumi})}
\item
  In Section 3, the techniques of different parts of Synopsis are
  proposed. But I feel the algorithms are not very clearly proposed. It
  is better to present the algorithms in the pseudo code manner.
\end{enumerate}

\begin{tcolorbox}
To address this point, we have added code listings for the calculation
of the fan-out score and our structural compaction algorithm in Section
3.2.2. For the distributed functionality (scaling in and out), a
previous iteration of the paper included pseudocode, but we found that
the sequence diagrams were more effective in communicating how the
protocols work. We also have added more detail to sections 3.2.1, 3.2.3,
and made a variety of tweaks to 3.3 and 3.4 to improve the clarity of
our algorithms.
\end{tcolorbox}

\begin{enumerate}
\def\labelenumi{\arabic{enumi})}
\setcounter{enumi}{1}
\item
  The theoretical performance analysis of the techniques is not
  discussed. Maybe it is better to propose some bounds of the techniques
  of Synopsis, e.g., bound of time complexity or bound of error.
\end{enumerate}

\begin{tcolorbox}
While our previous manuscript discussed the standard error reported
alongside query results, we now report the time complexity of both
insertions and lookups for the SIFT $(O(log\ n))$, as well as for our
quantization algorithm $(O(n))$. We also added a bit of discussion on how
the fan-out scores can be used to estimate memory bounds; briefly, given
fan-out scores for each level in the hierarchy, we can estimate the
total number of vertices, edges, and leaves that will exist in the SIFT.
This allows us to make accurate judgements about memory consumption in
the system.
\end{tcolorbox}

\begin{enumerate}
\def\labelenumi{\arabic{enumi})}
\setcounter{enumi}{2}
\item
  For experiments, just one dataset is used in experiments. More
  datasets should be introduced to study the performance of Synopsis
  extensively.
\end{enumerate}

\begin{tcolorbox}
Thank you for this suggestion! We have added a benchmark that
incorporates air quality data sourced from XXX to Section X.X. We feel
this addition substantially strengthens the claims made in the paper,
and also corroborates the results from our previous experiment with
atmospheric data -- both benchmarks demonstrate similar profiles in
memory consumption as new observations are assimilated. In general, this
helps illustrate how Synopsis can scale as additional data is stored,
while handling diversity in dataset types.
\end{tcolorbox}

\begin{enumerate}
\def\labelenumi{\arabic{enumi})}
\setcounter{enumi}{3}
\item
  Synopsis should be compared with existing works in experiments to more
  clearly show the advantages of Synopsis. It is better to pick the
  state-of-the-art existing technique and make comparison in
  experiments.
\end{enumerate}

\begin{tcolorbox}
We felt that Spark SQL would be a good point of comparison with
Synopsis; while an exact 1:1 system does not exist, Spark is currently
at the state-of-the-art in cluster computing and analysis. In this
benchmark, we compared query latency between the two systems. TODO more
\end{tcolorbox}

\begin{enumerate}
\def\labelenumi{\arabic{enumi})}
\setcounter{enumi}{4}
\item
  In Section 4.6, just the random query is studied. It is better to use
  some real queries.
\end{enumerate}

\begin{tcolorbox}
While the queries used for these benchmarks were indeed random, it is
worth noting that the parameterization of our query generation test
harness was based on real-world usage patterns. In other words, we
observed the ranges of values and combinations of features that were
frequently requested from users of our copy of the NOAA dataset stored
on a local file system. We also used several queries developed for the
visualizations in the paper as seed queries for the generator. We have
updated the description of this benchmark to better explain how the test
queries were sourced.
\end{tcolorbox}

\begin{enumerate}
\def\labelenumi{\arabic{enumi})}
\setcounter{enumi}{5}
\item
  The experiment of tuning the number of machines of the cluster should
  be added. It is an important experiment to show the scalability of a
  distributed technique.
\end{enumerate}

\begin{tcolorbox}
(response)
\end{tcolorbox}

\begin{enumerate}
\def\labelenumi{\arabic{enumi})}
\setcounter{enumi}{6}
\item
  I am confused by Table 4. I think using full data should be more
  accurate than using a subset of the full data. But, Table 4 shows that
  the RMSE of using full data is higher than the RMSEs of using 10\% and
  20\% of the full data.
\end{enumerate}

\begin{tcolorbox}
This is an excellent point, and we have augmented our discussion in
Section 5.2 to better explain this result. While there are not
substantial differences in the error reported for each dataset size (an
RMSE of about 6 Kelvin), we believe that the additional data points
available in the full-sized dataset may have led to over-fitting rather
than a boost in model performance. As a result, the smaller synthetic
samples are less susceptible to over-fitting and perform slightly
better.
\end{tcolorbox}

\subsection*{Reviewer: 2}\label{reviewer-2}

This manuscript focuses on the problem of processing queries over a
stream of spatio-temporal observational data. Each item in the stream
contains a geographical location, a timestamp, and a set of key-value
pairs. The authors develop a number of components to efficiently process
the queries. Experimental results demonstrate that their proposal is
capable of high efficiency. Overall, this is a good piece of research
work that studies an interesting and practical problem. But it can be
improved, especially in terms of presentation.

Strong points: 1. The problem of processing general queries over
spatio-temporal observational data is quite useful and practical; 2.
Based on the experimental results, the proposed solutions are efficient.

Weak points: 1. The presentation can be improved. The overflow of this
paper is not easy to follow; 2. The technical contributions and
challenges of the problem studied in this work are not presented in a
clear way.

\begin{tcolorbox}
(response)
\end{tcolorbox}

\begin{enumerate}
\def\labelenumi{(\arabic{enumi})}
\item
  In Section 1, the first and second items in ``challenges'' are
  similar. High data arrival rates will incur high data volumes.
  Consequently, challenges of ``data volumes'' and ``data arrival
  rates'' can be merged;
\end{enumerate}

\begin{tcolorbox}
Thank you for bringing this to our attention. We have merged the first
two challenges listed in Section 1.
\end{tcolorbox}

\begin{enumerate}
\def\labelenumi{(\arabic{enumi})}
\setcounter{enumi}{1}
\item
  The overflow is not easy to follow. The format/definition of the
  queries to be handled is still unclear. The problem definition (i.e.,
  the format of the observational streaming data, the definition of the
  queries, etc.) can be presented before Section 2;
\end{enumerate}

\begin{tcolorbox}
We appreciate this feedback and have taken a number of steps to improve
readability of the manuscript:

\begin{itemize}
\item
  Additional headings and formatting improvements (`System Components'
  in Sec. 1, indentation for the components, \ldots{})
\item
  Removal of some redundant/repetitive text (Figure captions that
  duplicated manuscript text, \ldots{})
\item
  item 3\ldots{}
\end{itemize}

We have also added a new `Problem Definition' section, placed before
Section 2 as suggested. We feel that this helps bring readers up to
speed with the trade-off space we are working with and also introduces
the kind of queries Synopsis supports early on. Thank you for this
suggestion!
\end{tcolorbox}

\begin{enumerate}
\def\labelenumi{(\arabic{enumi})}
\setcounter{enumi}{2}
\item
  In Section 3.3, what are the novelty and technical challenge(s) of
  ``sketchlet'', especially in the context of ``distributed
  maintenance''?
\end{enumerate}

\begin{tcolorbox}
(response)
\end{tcolorbox}

\subsection*{Reviewer: 3}\label{reviewer-3}

This paper proposes a distributed sketch over spatiotemporal streams
called SYNOPSIS. This sketch maintains a compact representation of the
streaming data, organized as a so-called SIFT structure, and it supports
dynamic scaling to preserve responsiveness and avoid overprovisioning. A
set of queries are supported by the proposed sketch, such as relational
queries, statistical queries, etc. The experimental study demonstrates
the efficacy of SYNOPSIS.

\begin{tcolorbox}
(response)
\end{tcolorbox}

My major concern is that the core technique of the sketch is based on
the previous work {[}11{]}, and a set of queries can be supported is
because of the usage of Welford's method {[}11{]}, and thus the novelty
is limited, although the authors take into account the varied data
density and arrival rates. I think the authors need to exploit more
novel techniques to support more types of queries (maybe in the future
work?).

\begin{tcolorbox}
Welford's method is indeed a key part of the functionality in Synopsis;
however, we feel that the true novelty of our approach is derived from
the hierarchical organization of metadata in the SIFT, which eventually
leads to these statical summaries. While other approaches may choose to
create a multidimensional matrix of feature values or store statistics
at each vertex in the graph, we were able to substantially reduce memory
consumption by tracking query contexts through the SIFT and storing data
solely at the leaves. Combined with our use of Geohashes, we can
effectively scale the SIFT in and out at runtime with changing resource
demands.

It is also worth noting that the primary benefit of using Welford's
method is that we can track distributions within metadata bins on a
fine-grained level; while it would certainly be possible to operate the
system without the presence of these statistical summaries (simply
storing bin counts instead), the accuracy of (1) query results, and (2)
synthetic datasets would decrease.

TODO discuss how we improved the methodology to highlight this.
\end{tcolorbox}

Another question is about the spatial or temporal query window size. The
distributed sketch is based on the geohash algorithm, which divides the
earth into a hierarchy of bounding boxes. The spatial range specified in
the query may not cover the bound boxes exactly, which means that only
part of the data in a box should be considered rather than the entire
data in the box. However, the record in each box used in query
processing contains statistics for the whole box. How can the accuracy
of the queries be guaranteed?

\begin{tcolorbox}
This is an excellent question, and we have updated the text (Section
X.X) to better explain how such spatial queries are resolved. In short,
we maintain a full-resolution geohash for each observation. While this
incurs additional memory overhead compared to simply not storing the
entire hash, it allows us to perform fine-grained queries. In situations
where multiple geohash bounding boxes overlap the query region, we
decompose the query into the smallest possible geohash ranges and then
eliminate any non-matching observations by calculating and comparing
their respective latitude/longitude points. As a result, spatial
retrievals in Synopsis are somewhat similar to an R-Tree {[}cite{]}
traversal.
\end{tcolorbox}

The SIFT structural compaction is not described clearly enough. I
suggest the authors add an example in that section. Is it true that it
does not matter how the original SIFT is constructed (either spatial
first level or temporal first level), because it will be reconfigured
dynamically?

\begin{tcolorbox}
That is correct: the SIFT will be reconfigured based on observed
distributions and fan-out scores. Along with the improvements suggested
by Reviewer 1, we have revamped Section 3.2.2 to be much more
straightforward. We observe the full-resolution values stored in the
SIFT, calculate fan-out scores for each feature (the average number of
outgoing edges for feature vertices), sort the features based on the
score (from low to high), and then reconfigure the tree hierarchy to
match the sorted feature fan-out scores.
\end{tcolorbox}

Yufei Tao et al. proposed a sketch-based method for spatio-temporal
aggregation (Spatio-Temporal Aggregation Using Sketches. ICDE'04), which
is relevant to this work. The authors should discuss this paper in the
related work section.

\begin{tcolorbox}
Thank you very much for bringing this to our attention; we have added a
discussion about this work and some of the related technologies it
builds on to the related work (Section 6).
\end{tcolorbox}

\end{document}
