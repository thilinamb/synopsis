\section{Related Work}
\label{sec:related}
Tao et al.~\cite{tao2004spatio} implements a sketch to answer distinct sum and distinct count queries for spatiotemporal data. They are using a sketch index similar to an aRB-tree~\cite{papadias2002indexing} where spatial indexing is implemented is using a R-tree and the temporal indexing is implemented as a B-tree. At the leaf nodes of the B-tree, a sketch implemented following the Flajolet-Martin algorithm~\cite{flajolet1985probabilistic} is used to capture an approximate view of the observations. This approach significantly reduces the space requirements for answering distinct sum/count queries on spatiotemporal data and provides efficient query evaluations mainly due to its ability to prune the sections of the search space.
Our work is different from theirs mainly because of its ability to capturing multiple features and their interactions facilitating a more broader set of queries.

Data~Cubes~\cite{gray1996data,harinarayan1996implementing,mumick1997maintenance,ho1997range} are a data structure for Online Analytical Processing that provide multidimensional query and summarization functionality. These structures generalize several operators provided by relational databases by projecting two-dimensional relational tables to N-dimensional cubes (also known as \emph{hypercubes} when $N > 3$). Variable resolution in Data Cubes is managed by the \emph{drill down/drill up} operators, and \emph{slices} or entire cubes can be summarized through the \emph{roll up} operator. While Data Cubes provide many of the same features supported by \textsc{Synopsis}, they are primarily intended for single-host offline or batch processing systems due to their compute- and data-intensive updates. In fact, many production deployments separate transaction processing and analytical processing systems, with updates pushed to the Data Cubes periodically. 

CHAOS~\cite{gupta2009chaos} builds on Data Cubes in a single-host streaming environment by pushing updates to its \emph{Computational Cubes} more frequently. To make dimensionality and storage requirements manageable, Computational Cubes only index summaries of incoming data that are generated during a preprocessing step. However, full-resolution data is still made available through continuous queries that act on variable-length sliding windows.
CHAOS builds its summaries using a wavelet-based approach, which tend to be highly problem-specific. Additionally, updates to the cube are still generated and published periodically rather than immediately as data is assimilated. 

Galileo~\cite{malensek2016analytic,malensek2015fast} is a distributed hash table that supports the storage and retrieval of multidimensional data. Given the overlap in problem domain, Galileo is faced with several of the same challenges as \textsc{Synopsis}. However, the avenues for overcoming these issues diverge significantly due to differences in storage philosophy: \textsc{Synopsis} maintains its dataset completely in main memory, avoiding the orders-of-magnitude disparity in I/O throughput associated with secondary storage systems. This makes \textsc{Synopsis} highly agile, allowing on-demand scaling to rapidly respond to changes in incoming load. Additionally, this constraint influenced the trade-off space involved when designing our algorithms, making careful and efficient memory management a priority while striving for high accuracy.

Simba (Spatial In-Memory Big data Analytics)~\cite{xiesimba} extends Spark SQL~\cite{armbrust2015spark} to support spatial operations in SQL as well as DataFrames. It relies on data being stored in Spark~\cite{zaharia2010spark}. Despite its higher accuracy, it is not scalable for geospatial streams in the long term due to high storage requirements. In \textsc{Synopsis}, spatial queries can be executed with a reasonable accuracy without having to store the streaming data as-is.

Dynamic scaling and elasticity in stream processing systems has been studied thoroughly \cite{heinze2014auto, gulisano2012streamcloud, castro2013integrating, loesing2012stormy, heinze2013elastic, schneider2009elastic}.
Heinze et al.~\cite{heinze2014auto} explores different dynamic scaling schemes including threshold-based rules and reinforcement learning using the FUGU~\cite{heinze2013elastic} stream processing engine.
Based on these schemes, the operators are continuously migrated between hosts in a FUGU cluster in order to optimize the resource utilization and to maintain low latency.
The approach is quite different from \textsc{Synopsis}, as we perform a targeted load migration where the workload of a computation is dynamically adjusted instead of entirely moving it to a host with a higher or lower capacity than the current host.
Further, we do not interrupt the data flow through \textsc{Synopsis} when dynamic scaling activities are in progress, whereas in FUGU the predecessor operator is temporarily paused until operator migration completes.

StreamCloud~\cite{gulisano2012streamcloud} relies on a global threshold-based scheme to implement elasticity where a query is partitioned into sub-queries which run on separate clusters.
It relies on a centralized component, the Elastic Manager, to initiate the elastic reconfiguration protocol, whereas in \textsc{Synopsis} each node independently initiates the dynamic scaling protocol.
This difference is mainly due to different optimization objectives of the two systems; StreamCloud tries to optimize the average CPU usage per cluster while \textsc{Synopsis} attempts to maintain stability at each node.
The state recreation protocol of StreamCloud is conceptually similar to our state transfer protocol, except that tuples are buffered at the new location until the state transfer is complete, whereas in \textsc{Synopsis} the new sketchlet immediately starts building the state which is later merged with the state (transferred asynchronously) from the parent sketchlet.

Gedik et al.~\cite{schneider2009elastic} also uses a threshold-based local scheme similar to \textsc{Synopsis}. Additionally, this approach keeps track of the past performance achieved at different operating conditions in order to avoid oscillations in scaling activities.
The use of consistent hashing at the splitters (similar to geohash based stream partitioning in \textsc{Synopsis}) achieves both load balancing and monotonicity (elastic scaling does not move states between nodes that are present before and after the scaling activity).
Similarly, our geohash-based partitioner together with control algorithms in \textsc{Synopsis} balance the workload by alleviating hotspots and sketchlets with lower resource utilization.
Our state migration scheme doesn't require migrating states between sketchlets that do not participate in the scaling activity, unlike with a reconfiguration of a regular hash-based partitioner.
Unlike in \textsc{Synopsis}, in their implementation, the stream data flow is paused until state migration is complete using vertical and horizontal barriers.
Finally, \textsc{Synopsis}' scaling schemes are placement-aware, meaning certain nodes are preferred when performing scaling with the objective of reducing the span of the distributed sketch.

