\section{Conclusions and Future Work}
\label{sec:conclusions}
Our framework for constructing a distributed sketch over spatiotemporal streams, \textsc{Synopsis}: (1) maintains a compact representation of the observational space, (2) supports dynamic scaling to preserve responsiveness and avoid overprovisioning, and (3) supports a rich set of queries to explore the observational space. Our methodology for achieving this is broadly applicable to other stream processing systems.  Our empirical benchmarks, with real-world observational data, demonstrate the suitability of our approach.

We achieve compactness in our sketchlet instances by dynamically managing the number of vertices in the SIFT hierarchy as well as the range each vertex is responsible for. We also maintain summary statistics and metadata within these vertices to track the distribution/dispersion of feature values and their frequencies. As a result, \textsc{Synopsis} is able to represent datasets using substantially less memory (\textbf{RQ-1}). Given variability in the rates and volumes of data arrivals from different geolocations, our scaling mechanism avoids overprovisioning and alleviates situations where sketch updates cannot keep pace with data arrival rates. Memory pressure is also taken into account during replica creation as well as when scaling in and out (\textbf{RQ-2}). During evaluations, only the sketchlets that hold portions of the observational space implicitly or explicitly targeted by the query are involved, ensuring high throughput. We support several high-level query operations allowing users to locate and manipulate data efficiently (\textbf{RQ-3}).

Our future work will target support for \textsc{Synopsis} to be used as input for long-running computations. Such jobs would execute periodically on a varying number of machines and could target the entire observational space or only the most recently-assimilated records. We also plan to implement continuous queries that can autonomously evolve with the feature space.
