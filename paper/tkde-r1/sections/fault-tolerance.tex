\textsc{Synopsis} relies on passive replication to recover from sketchlet failures because active replication increases resource consumption significantly and it is infeasible to use upstream backups because the state of a sketchlet depends on the entire set of messages it has processed previously \cite{castro2013integrating}.

Support for fault tolerance is implemented by augmenting the distributed sketch with a set of secondary sketchlets.
A sketchlet is assigned a set of $n$ secondary sketchlets on different machines to ensure \textsc{Synopsis} can withstand up to $n$ concurrent machine failures.
In our implementation, we used two secondary sketchlets ($n=2$) assigned to each sketchlet.
A primary sketchlet periodically sends the changes to its in-memory state as an \emph{edit stream} to its secondary sketchlets.
The secondary sketchlets, which act as the sink to the edit stream, serialize incoming messages to persistent storage.
This incremental checkpointing scheme consumes less bandwidth compared to a periodic checkpointing scheme that replicates the entire state \cite{castro2013integrating}.
By default, \textsc{Synopsis} uses the disk of the machine executing the secondary sketchlet as the persistent storage, but highly-available storage implementations such as HDFS~\cite{borthakur2008hdfs} can be used if necessary.
To reduce resource footprints, secondary sketchlets do not load the serialized state into memory unless they are promoted to being a primary.

System-wide incremental checkpoints are orchestrated by a special control message emitted by the stream ingesters.
\textsc{Synopsis} uses upstream backups at stream ingesters to keep a copy of the messages that entered the system since the last successful checkpoint.
In case of a failure, all messages since the last checkpoint will be replayed.
Sketchlets are implemented as idempotent data structures using message sequence numbers, hence they will process a replayed message only if it was not processed before.
The interval between incremental checkpoints can be configured based on time or the number of emitted messages.
Frequent checkpoints can incur high overhead, whereas longer periods between successive checkpoints consume more resources for upstream backups and require longer replay durations in case of a failure.

Membership management is implemented using Zookeeper~\cite{hunt2010zookeeper}, which is leveraged to detect failed sketchlets.
Upon receiving notification of a primary sketchlet failure, a secondary sketchlet assumes the role of primary through a leader election algorithm.
The secondary will start processing messages immediately and begins populating its state from persistent storage in the background.
Given this mode of operation, there may be a small window of time during which the correctness of queries are impacted.
This is rectified once the stored state is loaded to memory and the replay of the upstream backup is completed.
The SIFT's support for merge operations as well as its ability to correctly process out of order messages is useful during the failure recovery process.

As per our fault tolerance scheme, the \textit{total time to recover from the failure} ($T_{total}$) can be modeled as follows.
\begin{align*}
    T_{total} &= T_{d} + \max{(T_{l}, T_{r})}      
\end{align*}
where $T_{d}$ = \textit{time to detect a failure}, $T_{l}$ = \textit{time to load persisted state} and $T_{r}$ = \textit{replay time for messages at the upstream node}.

The time required to detect failures mainly depends on the session timeout value used by Zookeeper to detect lost members and the delay in propagating the notification to other members. With a $5s$ session timeout in an active cluster, we observed a mean notification propagation delay of $5.5s$ (std. dev. = $0.911s$, $95^{th}$ percentile = $6.000s$). Configuring a lower session timeout will increase the chance of false positives if sketchlets become non responsive for a while due to increased load or system activities such as garbage collection. The time required to load the persisted storage depends on the size of the serialized sketchlet; we benchmarked the time it takes to repopulate the state in all sketchlets after ingesting NOAA data for 2014. The mean state re-population time was recorded as $16.602s$ with std. dev. = $23.215s$ and $95^{th}$ \%ile = $70.877s$. Replay time mainly depends on the checkpointing interval as well as the message ingestion rate. With a checkpointing interval of 10000 messages, we experienced a mean value of $0.662s$ (std. dev. = $0.026s$, $95^{th}$ \%ile = $0.707s$) to replay an upstream buffer.
