\section{Performance Evaluation}
\label{sec:performance}

\subsection{Datasets and Experimental Setup}
Our subject dataset was sourced from the NOAA North American Mesoscale (NAM) Forecast System \cite{noaa_nam}.  The NAM collects atmospheric data several times per day and includes features of interest such as surface temperature, visibility, relative humidity, snow, and precipitation. Each observation in the dataset also incorporates a relevant geographical location and time of creation. This information is used during the data ingest process to partition streams across available computing resources and preserve temporal ordering of events.

Our performance evaluation was carried out on a 75-node heterogeneous cluster consisting of 45 HP DL160 servers (Xeon E5620, 12 GB RAM) and 30 HP DL320e servers (Xeon E3-1220 V2, 8 GB RAM) running Fedora 23. Rivulet was executed under the OpenJDK Java runtime, version 1.8.0\_72.

\subsection{Micro benchmarks}
\subsubsection{Dynamic Scaling}
We evaluated how Rivulet dynamically scales when the data ingestion rate is varied.
An artificial computation logic was used in place of regular Rivulet sketch update function to gain a better control over the throughtput of a Rivulet node.
The state maintained at nodes were minimum, hence there was no significant memory pressure.
The data ingestion rate was varied over time such that the peak data ingestion rate is less than the highest possible throughput which will create a backlog at Rivulet nodes.
We used the number of sketch instances created in the system to quantify the scaling activities.
If the system scales out, more sketch instances will be created in child nodes after the targeted load migration.
We started with a single Rivulet node and allowed the system to dynamically scale.
As it can be observed in Figure~\ref{fig:dyn-scaling}, the number of sketch instances vary with the ingestion rate.
Since we allow aggressive scale out, it shows a rapid scale out activity with high data ingestion rates whereas scaling in takes place gradually with one sub region (hence one sketch) at a time.
\begin{figure}
    \centerline{\includegraphics[width=3.5in]{figures/dyn-scaling.pdf}}
    \caption{The variation of number of sketch instances with the data ingestion rate.}
    \label{fig:dyn-scaling}
\end{figure}

\subsection{System Benchmarks}

