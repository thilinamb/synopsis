\section{Performance Evaluation}
\label{sec:performance}

\subsection{Datasets and Experimental Setup}
Our subject dataset was sourced from the NOAA North American Mesoscale (NAM) Forecast System \cite{noaa_nam}.  The NAM collects atmospheric data several times per day and includes features of interest such as surface temperature, visibility, relative humidity, snow, and precipitation. Each observation in the dataset also incorporates a relevant geographical location and time of creation. This information is used during the data ingest process to partition streams across available computing resources and preserve temporal ordering of events.

Our performance evaluation was carried out on a 75-node heterogeneous cluster consisting of 45 HP DL160 servers (Xeon E5620, 12 GB RAM) and 30 HP DL320e servers (Xeon E3-1220 V2, 8 GB RAM) running Fedora 23. \textsc{Synopsis} was executed under the OpenJDK Java runtime, version 1.8.0\_72.

\subsection{Microbenchmarks}
\subsubsection{Dynamic Scaling}
We evaluated how \textsc{Synopsis} dynamically scales when the data ingestion rate is varied.
An artificial computation logic was used in place of regular \textsc{Synopsis} sketch update function to gain a better control over the throughput of a \textsc{Synopsis} node.
The state maintained at nodes were minimum, hence there was no significant memory pressure.
The data ingestion rate was varied over time such that the peak data ingestion rate is less than the highest possible throughput which will create a backlog at \textsc{Synopsis} nodes.
We used the number of sketch instances created in the system to quantify the scaling activities.
If the system scales out, more sketch instances will be created in child nodes after the targeted load migration.
We started with a single \textsc{Synopsis} node and allowed the system to dynamically scale.
As it can be observed in Figure~\ref{fig:dyn-scaling}, the number of sketch instances vary with the ingestion rate.
Since we allow aggressive scale out, it shows a rapid scale out activity with high data ingestion rates whereas scaling in takes place gradually with one sub region (hence one sketch) at a time.
\begin{figure}
    \centerline{\includegraphics[width=3.5in]{figures/dyn-scaling.pdf}}
    \caption{The variation of number of sketch instances with the data ingestion rate.}
    \label{fig:dyn-scaling}
\end{figure}

\subsection{Stability at individual nodes}
The objective of this benchmark was to demonstrate how scaling out operations manage to maintain stability at each node under varying workload conditions.
The same setup as in previous micro benchmark was used, but the evaluation metrics captured are corresponding to an individual node instead of the entire system.
We captured how backlog length and throughput at an individual node vary with the input rate when dynamic scaling is enabled.
The \textsc{Synopsis} node that was considered for the experiment immediately received data from stream ingesters, hence the input rate observed at the node closely resembled the varying data ingestion rate.
As shown in Figure~\ref{fig:stability-backlog}, scaling out helps a node to keep up with the variations in the workload which in turn causes the backlog to stay within a safe range.
It also demonstrates the infrequent rapid scaling out activities and the continuous gradual scaling down activities as explained in section~\ref{subsec:scaling-out}.
\begin{figure}
    \centerline{\includegraphics[width=3.5in]{figures/stability_partial.pdf}}
    \caption{How scaling out enables maintaining stability at \textsc{Synopsis} nodes.}
    \label{fig:stability-backlog}
\end{figure}

\subsubsection{Sketch Query Evaluation}
\begin{table}[h!]
    \renewcommand{\arraystretch}{1.3}
    \caption{Query evaluation times for each query type.}
    \label{tbl:query-times}
    \begin{center}
        \begin{tabular}{|l|c|c|}
            \hline
            \textbf{Query Type}      & \textbf{Eval. (ms)} & \textbf{Std. Dev.} \\
            \hline
            Density                  & 0.007                    & 0.005 \\
            \hline
            Set Cardinality          & 0.154                    & 0.088 \\
            \hline
            Set Frequency            & 0.036                    & 0.019 \\
            \hline
            Set Membership           & 0.015                    & 0.009 \\
            \hline
            Statistics               & 0.002                    & 0.001 \\
            \hline
            SubSketch Stat. (5 km$^2$)& 46.357                  & 1.287 \\
            \hline
            Relational (5 km$^2$)    & 40.510                   & 6.937 \\
            \hline
            Relational (25 km$^2$)   & 47.619                   & 6.355 \\
            \hline
            Relational (800 km$^2$)  & 53.620                   & 6.818 \\
            \hline
        \end{tabular}
    \end{center}
\end{table}



\subsection{System Benchmarks}

