\section{Introduction}
\label{sec:introduction}
The proliferation of remote sensing equipment (such as radars and satellites), networked sensors, commercial mapping, location-based services, and sales tracking techniques have all resulted in the exponential growth of spatiotemporal data. Spatiotemporal data comprise observations where both the location and time of measurement are available in addition to \emph{features} of interest (such as humidity, air quality, disease prevalence, sales, etc). This information can be leveraged in several domains to inform decision making, scientific modeling, simulations, and resource allocation. Relevant domains include atmospheric science, epidemiology, environmental science, geosciences, smart-city settings, and commercial applications. In these settings, queries over the data must be \emph{expressive} to ensure efficient retrievals. Furthermore, query evaluations must be executed in real time with low latency, regardless of data volumes.

Spatiotemporal datasets are naturally multidimensional with multiple features of interest being reported/recorded continuously for a particular timestamp and geolocation. The values associated with these features are continually changing; in other words, the dataset \emph{feature space} is always evolving.  Queries specified over these datasets may have a wide range of characteristics encompassing the frequency at which they are evaluated and their spatiotemporal scope. The crux of this paper is to support query evaluations over continually-arriving observational data. We achieve this via construction of an in-memory distributed \emph{sketch} that maintains a compact representation of the data. The queries may be continuous or discrete, involve sliding windows, and encompass varying geospatial scopes. \\

\subsection{Challenges}
Support for real-time evaluation of queries --- discrete and continuous --- over a feature space that is continually evolving introduces unique challenges. These include:
\begin{itemize}
    \item   \emph{Data volumes:} It is infeasible to store all the observational data. This is especially true if the arrival rates outpace the rate at which data can be written to disk.
    \item   \emph{Data arrival rates:} The data may arrive continually and at high rates. Furthermore, this rate of arrivals may change over time.
    \item   \emph{Accuracy:} Queries specified by the user must be accurate, with appropriate error bounds or false positive probabilities included in the results.
    \item   \emph{Spatiotemporal characteristics:} Queries may target spatiotemporal properties. For example, a user may be interested in feature characteristics for a geospatial location at a particular daily interval over a chronological range (for example, 2:00--4:00 pm over 2--3 months).
\end{itemize}

\subsection{Research Questions}
The challenges associated with implementing this functionality led us to formulate the following research questions:
\begin{enumerate}
    \item   How can we generate compact, memory-resident representations of the observational space? These \emph{sketches} must be amenable to fast, continuous updates to ensure they are representative of the observational space.
    \item   How can we scale effectively in situations where the observations arrive at a rate faster than the rate at which the sketch can be updated? The density and arrival rates for observations vary based on the geospatial characteristics of the data. For example, in a smart-city setting, New York City would have a far higher rate of observations than Denver, Colorado.
    \item   How can we account for spatiotemporal attributes associated with observations and queries? This spans the generation and updates of the sketch, scaling in response to high data arrival rates, and the evaluation of queries.
    \item   How can we ensure low latency in the assimilation of observations, updates to the sketch, and query evaluations? A particular challenge is preservation of accuracy in query evaluations despite high data arrivals, scaling in response to changes in system load, and concurrent query evaluations. \\ \\ \vspace{2em}
\end{enumerate}

\subsection{Methodology}
Our methodology targets (1) creation of the distributed sketch, (2) updating the sketch in response to data arrivals, (3) splitting and fusing sketch instances to deal with hotspots, (4) using the distributed sketch to perform query evaluations, and (5) achieving low latency by minimizing synchronization between different components in the system. 

The sketch supports three key features: maintaining a compact representation of incoming data streams, splitting and fusing portions of the overall sketch, and expressive query evaluations. Relationships between observations and their values are maintained in a graph-based structure that employs several online, in-memory summarization techniques. The sketch is also naturally amenable to distribution, with each node in the cluster holding information about a particular subset of the observational space.  This ensures each node in the system can evaluate multiple concurrent queries independently.

Distributing the sketch across multiple nodes allows us to maintain a finer-grained representation of the feature space while also improving the accuracy of query evaluations; for example, an arctic region and a tropical region would be maintained on separate nodes that specialize for particular climates. Additionally, the sketch updates its precision dynamically to ensure representativeness of the dataset. This is achieved in one of two possible modes: biasing towards the recent past, or ensuring equal representation. In the former approach, the sketch targets finer-grained representations of observations in the recent past, and progressively coarser grained representations for the farther past. This scheme ensures that the apportioning of the available memory for the sketch is proportional to the time intervals under consideration.  In the latter approach, the sketch adaptively moves toward a coarser-grained representation as data accumulates.


\subsection{Paper Contributions}
In this paper, we present a framework for real-time query evaluations over voluminous, time-series data streams. Our specific contributions include:
\begin{itemize}
\item   Design of a sketch that maintains compact, distributed representations of the observational space with support for low-latency queries.

\item   Dynamic scaling of the sketch in response to data arrival rates, with upscaling operations that support targeted alleviation of hotspots. Our framework manages the complexity of identifying these hotspots, splitting portions of the sketch, and migrating the relevant subsets to nodes with higher capacity. Most importantly, the framework achieves this while maintaining acceptable levels of accuracy in query evaluations.

\item   Support for discrete and continuous query evaluations across the observational space. Query evaluations can target arbitrarily sized (chronological) window sizes and geospatial scopes.
\end{itemize}
To our knowledge, \textsc{Synopsis} is the first sketch designed specifically for geospatial observational data. We have validated the suitability of our approach through a comprehensive set of benchmarks with real observational data. 

\subsection{Paper Organization}
The remainder of this paper is organized as follows. Section~\ref{sec:system} describes the underlying technology \textsc{Synopsis} is based on, followed by our methodology in Section~\ref{sec:methodology}. Section~\ref{sec:performance} contains our performance evaluation of the system. Section~\ref{sec:related} discusses related approaches, and Section~\ref{sec:conclusions} concludes the paper and outlines our future research direction.

