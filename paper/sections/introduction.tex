\section{Introduction}
\label{sec:introduction}
The proliferation of remote sensing equipment (such as radars and satellites), networked sensors, commercial mapping, location-based services, and sales tracking techniques have all resulted in the exponential growth of spatiotemporal data. Spatiotemporal data comprise observations where both the location and time of measurement are available in addition to \emph{features} of interest (such as humidity, air quality, disease prevalence, sales, etc). This information can be leveraged in several domains to inform decision making, scientific modeling, simulations, and resource allocation. Relevant domains include atmospheric science, epidemiology, environmental science, geosciences, smart-city settings, and commercial applications. In these settings, queries over the data must be \emph{expressive} to ensure efficient retrievals. Furthermore, query evaluations must be executed in real time with low latency, regardless of data volumes.

Spatiotemporal datasets are naturally multidimensional with multiple features of interest being reported/recorded continuously for a particular timestamp and geolocation. The values associated with these features are continually changing; in other words, the dataset \emph{feature space} is always evolving.  Queries specified over these datasets may have a wide range of characteristics encompassing the frequency at which they are evaluated and their spatiotemporal scope. The crux of this paper is to support query evaluations over continually-arriving observational data. We achieve this via construction of an in-memory distributed \emph{sketch} that maintains a compact representation of the data.

\subsection{Challenges}
Support for real-time evaluation of queries --- discrete and continuous --- over a feature space that is continually evolving introduces unique challenges. These include:
\begin{itemize}
    \item   \emph{Data volumes:} It is infeasible to store all the observational data. This is especially true if the arrival rates outpace the rate at which data can be written to disk.
    \item   \emph{Data arrival rates:} The data may arrive continually and at high rates. Furthermore, this rate of arrivals may change over time.
    \item \emph{I/O Costs:} Memory accesses are 5-6 orders of magnitude faster than disk accesses. Given the data volumes, disk accesses during query evaluations are infeasible.
    \item   \emph{Accuracy:} Queries specified by the user must be accurate, with appropriate error bounds or false positive probabilities included in the results.
    \item   \emph{Spatiotemporal characteristics:} Queries may target spatiotemporal properties. For example, a user may be interested in feature characteristics for a geospatial location at a particular daily interval over a chronological range (for example, 2:00--4:00 pm over 2--3 months).
\end{itemize}

\subsection{Research Questions}
The challenges associated with implementing this functionality led us to formulate the following research questions:
\begin{itemize}
    \item[\textbf{RQ-1}]   How can we generate compact, memory-resident representations of the observational space while accounting for spatiotemporal attributes? The resulting \emph{sketch} must be amenable to fast, continuous updates to ensure its representativeness.
    \item[\textbf{RQ-2}]   How can we scale effectively in situations where system load is high or the observations arrive at a rate faster than the rate at which the sketch can be updated? The density and arrival rates for observations may vary based on geospatial characteristics. For example, in a smart-city setting, New York City would have a far higher rate of observations than Denver, Colorado.
    \item[\textbf{RQ-3}]   How can we enable expressive, low-latency queries over the distributed sketch while also maintaining accuracy?  Given that the sketch is a compact representation of the data, queries facilitate high-level analysis without requiring users to understand the underlying system implementation.
\end{itemize}

\subsection{Approach Summary}
Similar to other sketches, the design of \textsc{Synopsis} is guided by the functionality that we wish to support. Synopsis is designed to be a compact, effective surrogate for voluminous data that can interoperate and provide input data to general purpose computations expressed using popular analytic engines such as Spark \cite{zaharia2010spark,armbrust2015spark}, TensorFlow \cite{abadi2016tensorflow,tensorflow}, Hadoop \cite{hadoop,shvachko2010hadoop,borthakur2008hdfs}, and VW \cite{langford2007vowpal}.   Synopsis extracts metadata from observations and organizes this information as a graph to support relational queries targeting different portions of the feature space; we support selection, joins, aggregations, and sorting. The edges and vertices within this graph maintain inter-feature relationships, while leaves contain online, in-memory summary statistics and correlation information to support statistical queries and generation of synthetic datasets.  The number of edges at each level within the subgraphs corresponds to density-based dynamic binning of a particular feature to support error reduction during query evaluations.

Our sketch is also naturally amenable to distribution, with each node in the cluster holding information about a particular subset of the observational space.  This ensures each node in the system can evaluate multiple concurrent queries independently. The nodes are capable of scaling in or out depending on streaming ingress rates and memory footprints, with scale-out operations that support targeted alleviation of hotspots. Our framework manages the complexity of identifying these hotspots, splitting portions of the sketch, and migrating the relevant subsets to nodes with higher capacity. Distributing the sketch across multiple nodes allows us to maintain a finer-grained representation of the feature space while also improving the accuracy of query evaluations; for example, an arctic region and a tropical region would be maintained on separate nodes that specialize for particular climates.

To our knowledge, \textsc{Synopsis} is the first sketch designed specifically for spatiotemporal observational data. We have validated the suitability of our approach through a comprehensive set of benchmarks with real observational data. 

\subsection{Dataset and Experimental Setup}
Our subject dataset was sourced from the NOAA North American Mesoscale (NAM) Forecast System \cite{noaa_nam}.  The NAM collects atmospheric data several times per day and includes features of interest such as surface temperature, visibility, relative humidity, snow, and precipitation. Each observation in the dataset also incorporates a relevant geographical location and time of creation. This information is used during the data ingest process to partition streams across available computing resources and preserve temporal ordering of events. The size of the entire source dataset was 25 TB.

Performance evaluations throughout the paper were carried out on a cluster of 40 HP DL160 servers (Xeon E5620, 12 GB RAM). For our application benchmarks on Apache Spark (version 2.0.1 with HDFS 2.6.0), we used our baseline cluster of 40 machines as well as 30 HP DL320e servers (Xeon E3-1220 V2, 8 GB RAM) and 30 HP DL60 servers (Xeon E5-2620, 16 GB RAM). The test cluster was configured to run Fedora 24, and \textsc{Synopsis} was executed under the OpenJDK Java runtime 1.8.0\_72.

\subsection{Paper Organization}
The remainder of this paper is organized as follows. Section~\ref{sec:system} provides an overview of the system, followed by our methodology and performance evaluations in Section~\ref{sec:methodology}. Section~\ref{sec:applications} demonstrates applications of \textsc{Synopsis}, Section~\ref{sec:related} discusses related approaches, and Section~\ref{sec:conclusions} concludes the paper and outlines our future research direction.
