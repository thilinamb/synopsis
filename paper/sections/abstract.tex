\begin{abstract}
    Networked observational devices have proliferated in recent years, contributing to voluminous data streams from a variety of sources and problem domains. These streams often have a spatiotemporal aspect and include multidimensional \emph{features} of interest. Processing such data in an offline fashion using batch systems or data warehouses is costly from both a storage and computational standpoint, and in many situations the insights within the data streams are only useful if they are discovered in a timely manner.

    In this study, we propose \textsc{Synopsis}, a stream processing system that builds an online distributed \emph{sketch} of incoming observational data. The sketch summarizes feature values and inter-feature relationships in memory to facilitate real-time query evaluations. As the data streams evolve and user query patterns change, \textsc{Synopsis} performs targeted dynamic scaling to ensure high accuracy and low query latencies alongside effective resource utilization. We evaluate our system in the context of a real-world spatiotemporal dataset and demonstrate its efficacy in both scalability and query evaluations. \\ \\
\end{abstract}
