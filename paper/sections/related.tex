\section{Related Work}
\label{sec:related}

Galileo \cite{} is a distributed hash table that supports the storage and retrieval of multidimensional data. Given the overlap in problem domain, Galileo is faced with several of the same challenges as \textsc{Synopsis}. However, the avenues for overcoming these issues diverge significantly due to differences in storage philosophy: \textsc{Synopsis} maintains its dataset completely in main memory, avoiding the orders-of-magnitude disparity in I/O throughput associated with secondary storage systems. This makes \textsc{Synopsis} highly agile, allowing on-demand scaling to rapidly respond to changes in incoming load. Additionally, this constraint influenced the trade-off space involved when designing our algorithms, making careful and efficient memory management a priority while striving for high accuracy.

Dynamic scaling and elasticity in stream processing systems are studied before~\cite{heinze2014auto, gulisano2012streamcloud, castro2013integrating, loesing2012stormy, heinze2013elastic}.
Heinze et al.~\cite{heinze2014auto} have explored using different dynamic scaling schemes including threshold based rules and reinforcement learning using the FUGU~\cite{heinze2013elastic} stream processing engine.
Based on these schemes, the operators are continuously migrated between hosts in a FUGU cluster in order to optimize the resource utilization and to maintain a low latency.
Their approach is quite different from ours, because in \textsc{Synopsis} we perform a targeted load migration where the workload of a computation is dynamically adjusted instead of entirely moving it to a host with a higher or lower capacity than the current host.
Further we do not interrupt the data flow through \textsc{Synopsis} when dynamic scaling activities are in progress whereas in FUGU predecessor operator is temporarily paused until the operator migration is complete. sou