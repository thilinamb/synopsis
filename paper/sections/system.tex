\section{System Overview}
\label{sec:system}
\textsc{Synopsis} is a distributed sketch constructed over voluminous spatiotemporal data streams.
The system is composed of several hosts that form a cluster, with each participating host managing one or more \textsc{Synopsis} \emph{nodes}.
Each node contains a \emph{sketchlet}, which maintains multidimensional data points falling within its assigned geospatial scope(s).
The number of nodes that comprise the sketch varies dynamically as the sketch scales in or out to cope with data arrival rates and memory pressure.
%Each sketchlet is responsible for one or more geographical scopes and is implemented as a stateful stream processing node that can build and retain state over time.
A stream partitioning scheme, based on the Geohash algorithm (described in Section 3.2), is used to route packets to the appropriate sketchlet.
Sketchlets ingest stream packets and construct compact, in-memory representations of the observational data by extracting metadata from individual stream packets.
During dynamic scaling operations, the geographical extents managed by a sketch varies.

In addition to the data plane, \textsc{Synopsis} relies on a set of auxiliary services that are needed to construct, update, and maintain the sketch and also to adapt to changing system conditions as listed below.
\begin{description}[leftmargin=*]
	\item[Control plane:] The control plane is responsible for orchestrating control messages exchanged between \textsc{Synopsis} nodes as part of various distributed protocols such dynamic scaling.
    It is decoupled from the generic data plane to ensure higher priority and low latency processing without being affected by buffering delays and backpressure during stream processing.

	\item[Gossip subsystem:] While a majority of the \textsc{Synopsis} functionality relies on the local state constructed at a particular node, certain functionalities require an approximate global knowledge.
    For instance, each sketchlet maintains a Geohash prefix tree to assist in distributed query evaluations by forwarding queries to sketchlets that are responsible for particular geographical extents.
        In order to establish and maintain this global view of the entire system, sketchlets gossip about their state periodically (based on time intervals and the number of pending updates) as well as when a change in state occurs.
    \textsc{Synopsis} supports \emph{eventual consistency} with respect to these updates given their inherent propagation and convergence delays.

	\item[Querying subsystem:] The querying subsystem is responsible for the distributed evaluation of queries.
    This involves forwarding queries to relevant sketchlets; in some cases, multiple sketchlets may be involved based on the geographical scope of the query.

    \item[Monitoring subsystem:] Sketchlets comprising \textsc{Synopsis} are probed periodically to gather metrics that impact performance of the system.
    These include memory utilization and backlog information based on packet arrival rates and updates to the in-memory structures.
    This information is used for dynamic scaling recommendations as explained in in Section~\ref{subsec:scaling-out}.
\end{description}
