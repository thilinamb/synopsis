\subsection{Query Evaluations}
\label{subsec:query-eval}
\textsc{Synopsis} incorporates support for user-defined queries that are evaluated over the distributed sketch.  Queries can be specified by the user in a SQL-like format or with JSON-based key-value descriptions similar to GraphQL~\cite{graphql}. Exact-match, range-based, and summarization queries are all supported over spatiotemporal bounds and individual feature values. The following example depicts how SQL-like queries can be formulated for evaluation over the sketch.

\begin{lstlisting}[language=SQL,style=custompy]
    SELECT MEAN(precipitation), MAX(wind_speed)
    WHERE temperature > 20 AND temperature < 30
    AND humidity > .8 AND CORRELATION(
        cloud_cover, precipitation) < -0.25
\end{lstlisting}

Depending on scaling operations and the spatial scope of the queries, evaluations are carried out on one or more sketchlets. Information on the placement of sketchlets in the system and their corresponding feature scopes is maintained at each sketchlet in a geohash prefix tree, with changes propagated through the network in an eventually-consistent manner as data is ingested and scaling maneuvers occur.

The entry point for these queries, called the \emph{conduit}, may be any of the sketchlets comprising the distributed sketch. During query evaluations, the first step is to identify the set of sketchlets relevant to the query. The conduit consults its prefix tree to locate sketchlets based on spatial, chronological, and feature constraints specified by the user. For spatial constraints that overlap multiple geohashes, a set of \emph{minimum bounding geohashes} (MBG) is constructed based on query geometry. Points falling outside the MBG are trimmed at the individual sketchlet level, in a fashion similar to traversing an R-Tree \cite{guttman1984r}.  After this process is complete, the conduit forwards the queries on to the sketchlets for evaluation and supplies the client with a list of responding sketchlets.  As queries execute, results are streamed back and merged by the client API. This strategy ensures that I/O and processing activities are interleaved.

In some situations, a trade-off arises between false positives and false negatives when queries overlap but do not completely cover quantization bins. Users are made aware of this through the error bounds provided by \textsc{Synopsis}, but in cases where false positives are undesirable, partially-matching bins can be pruned from the results. This also affects spatial queries, albeit much less often; while high-resolution geohashes are stored in the SIFT, the encoding scheme is inherently lossy and may produce false positives depending on the configured resolution.

Our distributed prefix tree enables query evaluations during both scaling in and out. For instance, when a conduit attempts to forward a query to a child sketchlet that is undergoing a scale-in operation, the request will be redirected to the its parent sketchlet. This process can continue recursively up through the network, ensuring queries will reach their destination.

\subsubsection{Query Types Supported by \textsc{Synopsis}}
\begin{description}[leftmargin=*]
    \item[Relational Queries] describe the feature space in the context of the hierarchical trees within our SIFT data structure and may target ranges of values; e.g., ``What is the relationship between temperature and humidity during July in Alaska, when precipitation was greater than 1 cm?'' These queries return a subset of the overall sketch.
    % Subsketches may be manipulated and inspected on the client side, and then reused in subsequent queries to request more detail or broaden the query scope. Relational queries can optionally return statistical metadata stored in the leaf nodes of the tree; this is also supported by our statistical query functionality.

    \item[Statistical Queries] allow users to explore statistical properties of the observational space. For example, users can retrieve and contrast correlations between any two features at different geographic locations at the same time. Alternatively, queries may contrast correlations between different features at different time ranges at the same geographic location. These queries also support retrieval of the mean, standard deviation, and feature outliers based on Chebyshev's inequality \cite{knuth1968art}.

    \item[Density Queries] support analysis of the distribution of values associated with a feature over a particular spatiotemporal scope. These include kernel density estimations, estimating the probability of observing a particular, and determining the deciles and quartiles for the observed feature.% Kernel density estimations can request the function itself, an integral over a range of values, or the probability of a single value.

    \item[Set Queries] target identification of whether a particular combination of feature values was observed, estimating the cardinality of the dataset, and identifying the frequencies of the observations. Each type of set query requires a particular data structure, with instances created across configurable time bounds (for instance, every day). Set membership is determined using space-efficient bloom~filters~\cite{bloom1970space}, while cardinality queries are supported by the HyperLogLog~\cite{flajolet2007hyperloglog} algorithm.

%To determine set membership, we use Bloom filters may produce false positives, but never false negatives. Besides returning a \texttt{true} or \texttt{false} result to the user, membership queries also include the probability of the answer being a false positive.  Set HyperLogLog is able to estimate cardinality with high accuracy and low memory consumption. Finally, observation frequencies are provided by the count-min data structure \cite{cormode2005improved}. Count-min is structurally similar to a bloom filter, but can be used to estimate the frequency of values within a particular error band. Frequency queries are accompanied by their associated confidence intervals and relative error.

\item[Inferential Queries] enable spatiotemporal forecasts to be produced for a particular feature (or set of features). Discrete inferential queries leverage existing information in the distributed sketch to make predictions; aggregate metadata stored in the leaves of the tree can produce two-dimensional regression models that forecast new outcomes across each feature type when an independent variable of interest changes.

%In their continuous form, inferential queries are backed by machine learning models that are \emph{installed} for a particular time window and can be trained using either the sketch or new, full-resolution values as they arrive. Continuous inferential queries can stream predictions back to the client on a regular interval, or a subsequent query can reference a particular model and parameterize it to make a single prediction. Our current implementation of \textsc{Synopsis} supports multiple linear regression, but our machine learning interface allows new models to be plugged into the system at run time.

\item[Synthetic Data Queries] allow users generate representative datasets based on the distributions stored in the sketch and then stream them to client applications for analytics. The size of the dataset may also be specified; for instance, 10\% of the volume of the original data points.
\end{description}
%
\begin{table}
    \renewcommand{\arraystretch}{1.2}
    \caption{Local query evaluation times (1000 iterations). \vspace{-1em}}
    \label{tbl:query-times}
    \begin{center}
        \begin{tabular}{|l|c|c|}
            \hline
            \textbf{Query Type}      & \textbf{Mean (ms)} & \textbf{Std. Dev. (ms)} \\
            \hline
            Density                  & 0.007                    & 0.005 \\
            \hline
            Set Cardinality          & 0.154                    & 0.088 \\
            \hline
            Set Frequency            & 0.036                    & 0.019 \\
            \hline
            Set Membership           & 0.015                    & 0.009 \\
            \hline
            Statistical               & 0.002                    & 0.001 \\
            \hline
            \hline
            Tree Only (5 km)        & 46.357                   & 1.287 \\
            \hline
            Tree + Meta (5 km)      & 40.510                   & 6.937 \\
            \hline
            Tree + Meta (25 km)     & 47.619                   & 6.355 \\
            \hline
            Tree + Meta (800 km)    & 53.620                   & 6.818 \\
            \hline
        \end{tabular}
    \end{center}
    \vspace{-2em}
\end{table}
Table~\ref{tbl:query-times} outlines tree traversal times for query evaluations. These queries were separated into two groups: conventional lookups and tree retrievals. Conventional lookups include density queries, set queries, and statistical queries, while tree retrievals request targeted portions of the SIFT.  While conventional lookups do not return a tree structure to the client, they still require a tree traversal to resolve. In general, tree retrievals consume more processing time due to their serialization and I/O costs; however, it is worth noting that varying the geographical scope across sketchlet sizes from 5km to 800km did not result in a proportionate increase in processing time.% Overall, the sketch provides low-latency query evaluations for each query type.
